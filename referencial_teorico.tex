\chapter{REFERENCIAL TEÓRICO}
\label{cap:referencial teórico}

O Processamento de Línguas Naturais possui grande enfoque nos dias atuais em decorrência da enorme gama de aplicações possíveis dentro das mais diversas áreas. Ramificado em alguns tipos, PLN possui técnicas e dificuldades especificas de cada diferente grupo. Destaca-se para o presente projeto, o processamento de textos em língua natural apresenta dificuldade em comuns com as outras áreas do PLN, como a omissão de informações, um problema cujo sua solução inexiste ou possui um tratamento extremamente complexo e limitado [CITAÇÃO ARTIGO UGMAR]; reconhecimento da variabilidade linguística onde podem existir termos ou construções sintáticas não reconhecidas pelas regras do processador em questão; ambiguidade em construções sintáticas que podem por sua vez levar a uma errônea classificação gramatical, reconhecimento de padrões de ironia tal que a interpretação semântica de um corpus completo ou uma única sentença possa ser equivocado; problemas clássicos e já conhecidos pelos pesquisadores e desenvolvedores da área de PLN a respeito de termos ou construções sintáticas complexas de se classificar mediante aplicação de um algorítimos (como uma característica de sua definição, sequencia lógica de instruções atômicas, livres de qualquer tipo de livre interpretação ou intuição. [CITAÇÃO: LIVRO DO SANDERSON PAA]).

O uso de PLN para extração de informações a partir de documentos de texto, eleva as taxas de sucesso quanto a interpretação das informações extraídos, onde uma vez aplicada em tarefas que dependem de um certo grau de interpretação arbitrária de um ser humano, é capaz de reduzir a quantidade de equívocos decorrentes de tal interferência.

A revisão da literatura executada ao decorrer da pesquisa, mostrou que o poder do processamento de línguas naturais podem alcançar resultados satisfatórios ao reconhecer requisitos de um projeto e interpreta-los, como demonstrado na publicação Automated Service Selection Using Natural Language Processing [REFERENCIA], onde a extração e a interpretação de 28 requisitos e 91 descrições de serviços em texto de um cliente consumidor, a partir de 15 serviços previamente selecionados por uma pessoa, obteve-se uma taxa de 53\% de acerto nos resultados. Tal resultado, demonstra que a aplicação de algorítimos mais robustos para a extração de informações podem possuir relevância para o auxilio a atividades humanas.

O livro Introdução ao Teste de Software [REFERENCIA] proveu a base para o estado da arte relacionado ao tema de teste de software, onde foca-se o objetivo principal do sistema a ser construído, a partir de um documento de requisitos, a extração de casos de teste, além disso, diversos outros documentos científicos foram revisados objetivando a obtenção do conhecimento a cerca do processamento de línguas naturais, seus desafios e métodos de implementação. Também foi preciso muito pesquisa em torno do campo de estudo relacionado a engenharia de software. Todos esses conhecimentos compõe o estado da arte do projeto.