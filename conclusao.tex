\chapter{CONCLUSÃO}
\label{cap:conclusao}

Como supracitado na sessão de introdução do presente documento, essa etapa de pesquisa foi de enorme relevância para a decisão a respeito da viabilidade da aplicação de uma ferramenta de processamento de língua natural já existente sobre o projeto do sistema. Foram estudadas e testadas ferramentas que trabalham com o idioma Português brasileiro e Inglês, onde estas objetivaram o aproveitamento de seus mecanismos internos, onde talvez fossem necessário, como constatado nos capítulos anteriores, somente algumas alterações para integração com o sistema e construção dos módulos de suporte para o idioma Português.

Ao decorrer da pesquisa, foi adicionado a gama de possíveis ferramentas, o corretor gramatical Cogroo, desenvolvido para a suíte de aplicativos Libre Office e mantido sobre licença GNU/GPL - General Public License e código aberto. Cogroo deste então, devido a facilidade de integração com o sistema e sua robustez para com o tratamento de sentenças em língua natural sobre o idioma Português brasileiro, tornou-se o principal e escolhido candidato a ser implantado no sistema para a função de processamento da língua natural, fornecendo informações relevantes para a etapa posterior de extração da informação, com base na identificação das estruturas gramaticais.

A partir dos estudos sobre o core do código da ferramenta CoreNLP, principal candidata antes do Cogroo, constatou-se de que haverá necessidade da implementação de um módulo para analise do idioma Português brasileiro, contribuindo assim inclusive com o a evolução da ferramenta através da comunidade.

A partir da revisão bibliográfica da literatura que trata a respeito dos temas de processamento de línguas naturais e extração da informação, abstraiu-se a viabilidade de se desenvolver o presente sistema, extraindo relevantes informações a partir de um documento de requisitos e classificando com o fim de se identificar entidades, estruturas e casos de teste de um projeto de software. Verificou-se também os desafios dessa construções, em que tangem problemas relacionados ao processamento de línguas naturais e possíveis soluções para o mesmo.

Será necessário propor um modelo padronizado rigidamente de documentação de requisitos de software bem como avaliação das taxas de acerto da ferramenta candidata escolhida para corpus textuais do conteúdo relacionado ao que será fornecido para o programa.

Toda tal revisão da literatura foi crucial para o desenvolvimento dos conhecimentos na área de processamento de línguas naturais, teste de software e engenharia de software, todo o estado da arte que compõe o projeto e a conjunção de todas essas áreas de estudo.